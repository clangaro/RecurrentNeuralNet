\documentclass[11pt]{article}

\usepackage{amsmath, amssymb}
\usepackage{geometry}
\usepackage{setspace}
\usepackage{hyperref}

\geometry{margin=1in}
\onehalfspacing

\title{Stage 1 Formalisation: State Variables and Baseline Dynamics}
\author{}
\date{}

\begin{document}
\maketitle

\section{Purpose of Stage 1}

The goal of Stage~1 is to fully specify the dynamical system before introducing any learning rules.
This includes:
\begin{itemize}
    \item choosing explicit state variables,
    \item defining continuous-time equations of motion,
    \item fixing biologically reasonable parameter scales,
    \item ensuring the model admits intrinsic temporal structure without explicit delays or clocks.
\end{itemize}

At this stage, the network is untrained and serves as a baseline dynamical system.

\section{Network Architecture}

The model consists of three interacting populations:
\begin{itemize}
    \item an excitatory population $E_A$,
    \item an excitatory population $E_B$,
    \item an inhibitory population $I$.
\end{itemize}

The two excitatory populations are grouped together into a single excitatory vector
\[
\mathbf r_E(t) = 
\begin{bmatrix}
\mathbf r_A(t) \\
\mathbf r_B(t)
\end{bmatrix},
\]
while inhibitory activity is represented by $\mathbf r_I(t)$.

\section{State Variables}

The full dynamical state of the system is defined as:
\[
\mathbf x(t) = \big( \mathbf r_E(t),\ \mathbf r_I(t),\ \mathbf s_E(t) \big),
\]
where:
\begin{itemize}
    \item $\mathbf r_E(t) \in \mathbb{R}^{N_E}$ are excitatory firing rates,
    \item $\mathbf r_I(t) \in \mathbb{R}^{N_I}$ are inhibitory firing rates,
    \item $\mathbf s_E(t) \in \mathbb{R}^{N_E}$ is a filtered excitatory synaptic trace.
\end{itemize}

The synaptic trace $\mathbf s_E$ represents slow excitatory synaptic dynamics and provides an intrinsic temporal timescale.

\section{Synaptic Trace Dynamics}

Excitatory synaptic activity is modelled as a low-pass filtered version of excitatory firing rates:
\begin{equation}
\tau_s \frac{d\mathbf s_E}{dt}
=
-\mathbf s_E + \mathbf r_E.
\end{equation}

This equation arises from standard exponential synaptic filtering and introduces temporal persistence without delay lines.

\section{Excitatory Population Dynamics}

Excitatory firing rates evolve according to:
\begin{equation}
\tau_E \frac{d\mathbf r_E}{dt}
=
-\mathbf r_E
+
\phi\!\left(
W_{EE}\mathbf s_E
-
W_{EI}\mathbf r_I
+
\mathbf u_E(t)
+
\mathbf b_E
\right),
\end{equation}
where:
\begin{itemize}
    \item $W_{EE}$ denotes excitatory-to-excitatory synaptic weights,
    \item $W_{EI}$ denotes inhibitory-to-excitatory weights,
    \item $\mathbf u_E(t)$ is external excitatory input,
    \item $\mathbf b_E$ is a constant bias term,
    \item $\phi(\cdot)$ is a non-negative static transfer function.
\end{itemize}

Using $\mathbf s_E$ rather than $\mathbf r_E$ in recurrent excitation introduces slow temporal structure.

\section{Inhibitory Population Dynamics}

Inhibitory firing rates follow:
\begin{equation}
\tau_I \frac{d\mathbf r_I}{dt}
=
-\mathbf r_I
+
\phi_I\!\left(
W_{IE}\mathbf r_E
-
W_{II}\mathbf r_I
+
\mathbf u_I(t)
+
\mathbf b_I
\right),
\end{equation}
where $W_{IE}$ and $W_{II}$ are excitatory-to-inhibitory and inhibitory-to-inhibitory weight matrices, respectively.

Inhibitory dynamics are typically faster than excitatory dynamics.

\section{Nonlinearity}

Both populations use a threshold-linear transfer function:
\[
\phi(x) = \max(0, x).
\]
This choice enforces non-negative firing rates while preserving interpretability and analytical tractability.

\section{Parameter Scales}

The following biologically reasonable parameter values are fixed for Stage~1:
\begin{itemize}
    \item Excitatory time constant: $\tau_E = 20~\mathrm{ms}$,
    \item Inhibitory time constant: $\tau_I = 10~\mathrm{ms}$,
    \item Synaptic trace time constant: $\tau_s = 200~\mathrm{ms}$.
\end{itemize}

The separation $\tau_I < \tau_E \ll \tau_s$ ensures fast stabilisation by inhibition and slower temporal integration by recurrent excitation.

\section{Summary}

At the end of Stage~1, the model is a fully specified continuous-time dynamical system with:
\begin{itemize}
    \item no explicit clocks or delay lines,
    \item intrinsic temporal structure arising from synaptic dynamics,
    \item a clear separation of excitatory and inhibitory roles.
\end{itemize}

This system forms the baseline upon which learning rules and prediction-error phenomena will later be built.

\end{document}
